\Chapter{What is XGAP?}

In this chapter you find the answer to the above question beginning from a
short overview up to a description of the technical concept.

\Section{Basics}

The idea of \XGAP~is that \GAP~should be able to control graphics. A
graphical user interface is sometimes easier to use than a text and
command oriented one and there are mathematical applications for which it
can be quite useful to visualize objects with computer graphics.

On the other hand it is not sensible to change the whole concept and user
interface of \GAP~because people are used to it and it works. So \XGAP~is a 
separate C program running under the X Window System, which starts up a
\GAP~job and allows normal command execution within a window. In addition
there is a library written in \GAP, which makes it possible to open new
windows, display graphics, control menus and do other graphical user
communication in \GAP~via the separate C part.

Built on those ``simple'' windows and graphic objects are other libraries
which display graphs and posets in a window and allow the user to move
vertices around, select them and invoke \GAP~functions on mathematical
objects which belong to the graphic objects.

One ``application'' of these libraries is a program to display subgroup
lattices interactively. So \XGAP~works as a front end for mathematical
operations on subgroup lattices. It is possible to ``switch'' between the
graphics and the \GAP~commands. This means that you can for example use the 
graphically selected vertices resp. subgroups to do your own calculations
in the command window. You can then display your results again as vertices
in the lattice.

Of course there are other applications possible and the libraries are
developed with code reusage in mind.


\Section{What you can do with XGAP}


\Section{How does it work?}


